% Load the kaobook class
\documentclass[
    a4paper, % Page size
    fontsize=10pt, % Base font size
    twoside=true, % Use different layouts for even and odd pages (in particular, if twoside=true, the margin column will be always on the outside)
    %open=any, % If twoside=true, uncomment this to force new chapters to start on any page, not only on right (odd) pages
    %chapterentrydots=true, % Uncomment to output dots from the chapter name to the page number in the table of contents
    numbers=noenddot, % Comment to output dots after chapter numbers; the most common values for this option are: enddot, noenddot and auto (see the KOMAScript documentation for an in-depth explanation)
    fontmethod=tex, % Can also use "modern" with XeLaTeX or LuaTex; "tex" is the default for PdfLaTex, and "modern" is the default for those two.
]{kaobook}

% Choose the language
\ifxetexorluatex
    \usepackage{polyglossia}
    \setmainlanguage{english}
\else
    \usepackage[english]{babel} % Load characters and hyphenation
\fi
\usepackage[english=american]{csquotes}  % English quotes

% Load packages for testing
\usepackage{blindtext}
\usepackage{orcidlink}
\usepackage{siunitx}
\usepackage{astro}
% \usepackage{showframe} % Uncomment to show boxes around the text area, margin, header and footer
%\usepackage{showlabels} % Uncomment to output the content of \label commands to the document where they are used

% Load the bibliography package
\usepackage[backend=biber, style=numeric-comp,backref, sorting=none, firstinits=true]{biblatex} 
\DeclareSourcemap{
 \maps[datatype=bibtex,overwrite=true]{
  \map{
    \step[fieldsource=Collaboration, final=true]
    \step[fieldset=usera, origfieldval, final=true]
  }
 }
}

\renewbibmacro*{author}{%
  \iffieldundef{usera}{%
    \printnames{author}%
  }{%
    \printfield{usera}, \printnames{author}%
  }%
}%

\usepackage{kaobiblio}
% \bibliographystyle{apsrev4-2}
\addbibresource{thesis.bib} % Bibliography file

% Load mathematical packages for theorems and related environments
\usepackage[framed=true]{kaotheorems}

% Load the package for hyperreferences
\usepackage{kaorefs}

\graphicspath{{figures/}{images/}} % Paths where images are looked for

\makeindex[columns=3, title=Alphabetical Index, intoc] % Make LaTeX produce the files required to compile the index


\begin{document}

%----------------------------------------------------------------------------------------
%   BOOK INFORMATION
%----------------------------------------------------------------------------------------
\subject{Doctoral Thesis}
% \titlehead{Optical Follow-Up of High-Energy Neutrinos}
\title[Optical Follow-Up of High-Energy Neutrinos]{Optical Follow-Up of High-Energy Neutrinos}
\author[SR]{Simeon Reusch}% \orcidlink{0000-0002-7788-628X}}
\date{\today}
\publishers{Humboldt-Universität zu Berlin}

%----------------------------------------------------------------------------------------

\frontmatter % Denotes the start of the pre-document content, uses roman numerals

%----------------------------------------------------------------------------------------
%   COPYRIGHT PAGE
%----------------------------------------------------------------------------------------

\makeatletter
\uppertitleback{\@titlehead} % Header

\lowertitleback{
    % \textbf{Disclaimer} \\
    % You can edit this page to suit your needs. For instance, here we have a no copyright statement, a colophon and some other information. This page is based on the corresponding page of Ken Arroyo Ohori's thesis, with minimal changes.
    
    % \medskip
    
    \textbf{Copyright} \\
    \cczero\ This book is released into the public domain using the CC0 code. To the extent possible under law, I waive all copyright and related or neighbouring rights to this work. To view a copy of the CC0 code, visit: \\\url{http://creativecommons.org/publicdomain/zero/1.0}
    
    \medskip
 
    This thesis was typeset with the help of \href{https://sourceforge.net/projects/koma-script}{\KOMAScript} and \href{https://www.latex-project.org}{\LaTeX} using the \href{https://github.com/fmarotta/kaobook}{kaobook} class.
    
    \medskip

    The code used to typeset this thesis and create the figures within can be accessed at \href{https://github.com/simeonreusch/koma-thesis}{github.com/simeonreusch/thesis}

    \medskip
    
    \textbf{Publisher} \\
    First published in August 2023 by \@publishers
}
\makeatother


\maketitle

%----------------------------------------------------------------------------------------
%   PREFACE
%----------------------------------------------------------------------------------------

% \chapter*{Preface}

% \blindtext

%----------------------------------------------------------------------------------------
%   TABLE OF CONTENTS & LIST OF FIGURES/TABLES
%----------------------------------------------------------------------------------------

\begingroup % Local scope for the following commands

% Define the style for the TOC, LOF, and LOT
%\setstretch{1} % Uncomment to modify line spacing in the ToC
%\hypersetup{linkcolor=blue} % Uncomment to set the colour of links in the ToC
\setlength{\textheight}{230\vscale} % Manually adjust the height of the ToC pages

% Turn on compatibility mode for the etoc package
%\etocstandarddisplaystyle % "toc display" as if etoc was not loaded
%\etocstandardlines % "toc lines as if etoc was not loaded

\tableofcontents % Output the table of contents

\listoffigures % Output the list of figures

% Comment both of the following lines to have the LOF and the LOT on different pages
\let\cleardoublepage\bigskip
\let\clearpage\bigskip

\listoftables % Output the list of tables

\endgroup

\mainmatter
\setchapterstyle{kao}

%\chapter{Theoretical background}
%\addpart{Title of the Part}
\pagelayout{margin} % Restore margins
\setchapterimage[7cm]{ic_icecube.jpg}
\chapter{The IceCube Detector}
\labch{IceCube}
One of the two most relevant instruments for this thesis is the \textit{IceCube Detector} located at the geographic South Pole.

\setchapterimage[7cm]{ztf_telescope.png} % Optionally specify the height
\setchapterpreamble[u]{\margintoc}
\chapter{The Zwicky Transient Facility}
\labch{ZTF}
The second instrument relevant for this thesis is the Zwicky Transient Facility (ZTF). Named after the notorious Swiss-American astronomer Fritz Zwicky, it is a wide-field optical survey telescope located at Mount Palomar in California, United States, at 1700 m above sea level.\marginnote{See \url{https://sites.astro.caltech.edu/palomar/about/telescopes/oschin.html} for a historical overview.} The housing, the 1.22 m (48 inch) Samuel Oschin telescope, is a Schmidt design and was inaugurated in 1948 \cite{Harrington1952}. Originally, the telescope used photographic plates. As these have obvious drawbacks, and because technological progress made it possible, the Near-Earth Asteroid Tracking (NEAT) program \cite{Pravdo1999} replaced the photographic plates with a charge-coupled device (CCD) camera in the early 2000s.

The camera was updated several times over the course of the next years. The immediate predecessor of ZTF, the Palomar Transient Factory (PTF) \cite{Law2009}, began operation in 2009. Equipped with a 96 Megapixel camera, it already had many of the characteristics of ZTF: A fully automated survey, searching for optical transients.

PTF's successor in spirit, ZTF, was the first electronic camera using almost the full field of view (FOV) of the P48. The main design metric for ZTF was \textit{volumentric survey speed} \cite{Bellm2016}. This is the volume within which an object of given absolute magnitude can be detected in one exposure, divided by the total time for the exposure (observation plus overhead). The system saw first light in 2017, and started its scientific use in the year after. As of writing, it is still operational.

\section{Telescope Design}
A Schmidt telescope is 

\section{Camera}

\section{Image Processing}



%\chapter{The ZTF neutrino follow-up program}
%\chapter{Candidate TDE AT2019fdr: a possible source?}
%\chapter{The ZTF nuclear sample}
%\chapter{Conclusion and Outlook}
\appendix

\pagelayout{wide} % No margins
\addpart{Appendix}
\pagelayout{margin} % Restore margins


%----------------------------------------------------------------------------------------

\backmatter % Denotes the end of the main document content
\setchapterstyle{plain} % Output plain chapters from this point onwards

%----------------------------------------------------------------------------------------
%   BIBLIOGRAPHY
%----------------------------------------------------------------------------------------

% The bibliography needs to be compiled with biber using your LaTeX editor, or on the command line with 'biber main' from the template directory
%\defbibnote{bibnote}{Here are the references in citation order.\par\bigskip} % Prepend this text to the bibliography
\printbibliography[heading=bibintoc, title=Bibliography] % Add the bibliography heading to the ToC, set the title of the bibliography and output the bibliography note

%----------------------------------------------------------------------------------------
%   INDEX
%----------------------------------------------------------------------------------------

% The index needs to be compiled on the command line with 'makeindex main' from the template directory

\printindex % Output the index

\end{document}
