% \setchapterimage[7cm]{introduction/nebra_cutout.jpg}
\chapter*{Abstract}\label{abstract}
\markboth{Abstract}{Abstract}
\addcontentsline{toc}{chapter}{Abstract}
\labch{abstract}

% Thesis "Optical Follow-Up of High-Energy Neutrinos" ­– Abstract

This thesis is concerned with the origin of high-energy neutrinos detected by the IceCube Observatory at the South Pole. It summarizes the young field of neutrino astronomy and details the systematic high-energy neutrino follow-up program we have been conducting with the Zwicky Transient Facility (ZTF), an optical survey telescope, since four years.

One major finding is the establishment of the astrophysical transient AT2019fdr as counterpart to a high-energy neutrino. This transient is studied in detail, aided by the collection and reduction of data all across the electromagnetic spectrum. Based on these observations and modeling of the light curve I conclude that this event constitutes a Tidal Disruption Event (TDE), albeit an unusual one. The chance coincidence of such an association is \SI{0.034}{\percent} when including another previous association. Furthermore, I discuss the infrared dust echo from this transient in the context of two further possible associations of candidate TDEs with high-energy neutrinos, which also display such a dust echo.

This study is appended by the creation of the ZTF nuclear sample, the first systematic sample of transient events found near the cores of their host galaxies within the ZTF survey. One goal of this sample is to enlargen the number of TDEs found so far. This is achieved by the development of a machine-learning based photometric typing algorithm. That classifier is trained with a survey of bright ZTF transients, including augmentation of that sample to account for the fainter nature of the nuclear sample. When applying that classifier to the nuclear sample, an additional 27 new candidate TDEs are found. Furthermore, identification of candidate TDEs via their infrared dust-echo signal is also successful, resulting in 16 previously unpublished TDE candidates.

\newpage

Diese Dissertation befasst sich mit dem Ursprung der hochenergetischen Neutrinos, welche das IceCube-Observatorium am Südpol detektiert. Sie stellt das junge Feld der Neutrino-Astronomie vor und beschreibt das systematische Follow-Up-Programm für hochenergetische Neutrinos, das wir seit vier Jahren mit der Zwicky Transient Facility (ZTF) durchführen, einem optischen Teleskop.

Ein wesentliches Resultat ist die Identifikation des astrophysikalischen Objekts AT2019fdr als mögliche Quelle eines hochenergetischen Neurinos. Dieses Objekt wird im Detail untersucht; so werden Daten quer durch das elektromagnetische Spektrum zusammengetragen und analysiert. Basierend auf diesen Beobachtungen und einer Modellierung der Lichtkurve komme ich zu dem Schluss, dass AT2019fdr ein sogenanntes Tidal Disruption Event darstellt, wenn auch ein ungewöhnliches. Die Wahrscheinlichkeit, dass eine solche Assoziation nur Zufall ist, liegt bei \SI{0.034}{\percent}, wenn man eine weitere TDE-Neutrino-Assoziation mit einberechnet. Weiterhin diskutiere ich das Infrarot-Staubecho von diesem Objekt im Zusammenhang mit zwei weiteren Assoziationen von möglichen TDEs mit hochenergetischen Neutrinos, die ebenfalls ein solches Staubecho aufweisen.

Diese Studie wird begleitet von der Erstellung des ZTF nuclear sample, der ersten systematischen Sammlung innerhalb des ZTF-Datensatzes von solchen astrophysikalischen Ereignissen, die sich nahe dem Nukleus ihrer Wirtsgalaxie ereignen. Eines der Ziele dieser Untersuchung ist es, die Zahl der TDEs zu vergrößern. Ich bewerkstellige dies durch die Entwicklung eines Algorithmus mit Verfahren des maschinellen Lernens zur photometrischen Typisierung astrophysikalischer Ereignisse. Dieser Klassifikator wird mit einem Datensatz nahegelegener astrophysikalischer Ereignisse trainiert, der zusätzlich künstlich verrauschter und lichtärmer gemacht wird, um dem nuclear sample mehr zu entsprechen. In Anwendung dieses Klassifikators auf das nuclear sample finde ich 27 neue TDE-Kandidaten. Die Identifikation von TDE-Kandidaten mittels ihres Infrarot-Staubechos ist ebenfalls erfolgreich und resultiert in 16 bisher nicht publizierten TDE-Kandidaten.