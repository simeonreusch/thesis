\setchapterimage[7cm]{introduction/nebra_cutout.jpg}
\chapter*{Introduction}\label{introduction}
\labch{introduction}
For\marginnote{Sky disk of Nebra. With an age of roughly 4000 years, this bronze disk is the oldest known depiction of the night sky. The group of seven stars has been interpreted as the Pleiades. Image source: \href{https://commons.wikimedia.org/wiki/File:1600_Himmelsscheibe_von_Nebra_sky_disk_anagoria.jpg}{Wikimedia Commons}.} millennia, astronomy exclusively belonged to the domain of optics. Until the \nth{17} century, the unaided eye, sensitive to wavelengths from \num{380} to \SI{800}{\nano\m}, was the sole `instrument'~\sidecite{Wall2018} in observing the sky. Astronomy started with the practice of \textbf{observing and predicting the motions of the sun} to forecast seasons in ancient Babylon and Egypt\sidenote{In Egypt, the seasons were not particularly prominent. Therefore, the practice of tracking the sun relative to Sirius --- the brightest star visible --- to predict the Nile flooding had a direct impact on people's livelihood.}~\sidecite{Linton2004}. Over the centuries, observing the sky evolved into a mature science, leading to the Copernican Revolution, in which the sun replaced the Earth as the center of the solar system. This pre-instrumental period also already saw the \textbf{observation of galactic transient events}, like the supernova (SN) \emph{SN185} observed by Chinese astronomers almost two millennia ago, or the widely observed \emph{SN1054}.

Astronomy was greatly accelerated by the \textbf{invention of the telescope} by Hans Lippershey, which allowed for the observation of much fainter objects~\sidecite{Beckman2021}. Two important developments in the \nth{19} century further advanced the field: \textbf{Spectroscopy}, allowing the identification of chemical elements in extraterrestrial objects, and \textbf{photography}, which vastly increased the depth of observations by gathering much more light as is possible when using a telescope with an eyepiece.

\textbf{Radio and infrared astronomy} extended the observational window to higher wavelengths in the 1930s and 1950s. To peek into the high-frequency domain with \textbf{UV}, \textbf{X-ray} and \textbf{gamma-ray astronomy} was technologically even more challenging. It required satellites to escape the absorption of high-energy (HE) photons by the Earth's atmosphere, or --- for the most energetic photons --- required the development of ground-based Imaging Atmospheric Cherenkov Telescopes (IACTs) to observe very-high-energy gamma rays.

After the invention of the telescope, it took another 360 years for the photon to lose its supremacy as the only messenger around. The advent of \textbf{neutrino astronomy} in the 1960s and lastly, \textbf{gravitational wave astronomy} in the 2010s initiated the era of \textbf{multimessenger astronomy}.

Besides those three messengers, the early \nth{20} century saw the identification of a fourth potential messenger: \textbf{Cosmic rays}. These predominantly consist of highly energetic protons, electrons and helium nuclei. Cosmic rays come with a large drawback, though: Due to their charge, the particles constituting cosmic rays are deflected by magnetic fields on their way through interstellar and intergalactic space, which makes pin-pointing their origin a very hard task~\cite{Beckman2021}.

High-energy astrophyscial neutrinos could come to the rescue, as they are thought to stem from the very same processes that create cosmic rays. Their existence has been firmly established a decade ago, but their sources remain --- at least partially --- a mystery.

This thesis is concerned with