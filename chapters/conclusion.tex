\setchapterimage[7cm]{conclusion/composite_crop.png}
\chapter{Conclusion and Outlook}\label{conclusion}
\labch{conclusion}
The\marginnote{Left: The site of the Rubin Observatory in Chile (image credit: Rubin Observatory, NSF and AURA). Right: IceCube Gen2 (image credit: DESY and Science Communication Lab).} young field of \textbf{neutrino astronomy} is still in flux, with only a few established sources of high-energy neutrinos so far. This work gave an introduction into the field, the instruments involved and the \textbf{high-energy neutrino follow-up program}. The latter has now been running for over four years with significant contribution by the author, yielding several sources coincident with high-energy neutrinos. One of these, \textbf{candidate Tidal Disruption Event \textit{AT2019fdr}} has been analyzed in detail in this work, including the compilation of an extensive set of multi-wavelength data from various instruments, modeling of its light curve, a study of its dust echo properties and a discussion of the chance coincidence of such an association.

This analysis has been supplemented by the \textbf{ZTF Nuclear Sample}, created by the author of this work. It comprises by a unique dataset of nuclear transients observed with high-cadence in optical wavelengths over a period of several years\todo{does it though?}. The sample has been tentatively classified with machine learning methods, utilizing an augmented version of the Bright Transient Survey as training sample. This classification, aided by an iterative visual selection of the `TDE region', yielded a list of new candidate TDEs.\todo{we need that list}

There are various further studies that could be done with the Nuclear Sample:

So far, we do not know for sure if Tidal Disruption Events or similar violent accretion events do in fact produce high-energy neutrinos. The associations of three events so far do stipulate that, but hinge on the correctness of the error regions published by IceCube. Also, since roughly 12 months the follow-up program has not yielded any promising candidates. Ultimately, only time will tell.

Two important new instruments will allow to shed more light on the origin of high-energy neutrinos. First and foremost, IceCube-Gen2 \sidecite{Aartsen2021} is underway, with a goal to be fully operational in the middle of the 2030s. This extended detector will most likely improve upon the localization accuracy of IceCube, rendering associations with sources more secure.