\setchapterimage[7cm]{conclusion/composite.jpg}
\chapter*{Conclusion and Outlook}\label{conclusion}
\markboth{Conclusion and Outlook}{Conclusion and Outlook}
\addcontentsline{toc}{chapter}{Conclusion and Outlook}
\labch{conclusion}
The\marginnote{The site of the Rubin Observatory in Chile. Image credit: Rubin Observatory, NSF and AURA.} young field of \textbf{neutrino astronomy} is still in flux, with only a few established sources of high-energy neutrinos so far. This work gave an introduction into the field, an overview over predicted source classes, the instruments involved and the procedures and results of the \textbf{high-energy neutrino follow-up program}. The latter has now been running for over four years with significant contribution by the author, yielding several sources coincident with high-energy neutrinos. One of these, \textbf{candidate Tidal Disruption Event \textit{AT2019fdr}} has been analyzed in detail in this work, including the compilation of an extensive set of multi-wavelength data from various instruments, modeling of its light curve, a study of its dust echo properties and a discussion of the chance coincidence of such an association.

This analysis has been supplemented by the \textbf{ZTF Nuclear Sample}, created by the author of this work. It comprises by a unique dataset of nuclear transients observed with high-cadence in optical wavelengths over a period of several years. The sample has been tentatively classified with machine learning methods, utilizing an augmented version of the Bright Transient Survey as training sample. This classification, aided by an iterative visual selection of the `TDE region', yielded 27 new candidate TDEs. Furthermore, a selection based on infrared dust echoes resulted in 16 previously unpublished TDE and accretion flare candidates (2 of which are also selected by the classifier). In total, \textbf{41 new TDE candidates} were identified in this work.

% It highlighted two immediate avenues for further research: Establishing a realtime program making use of potential dust echoes as a tracer

\subsubsection{Black Hole Masses for the Nuclear Sample}
There are various further studies that could be done with the nuclear sample. The nature of accretion flares like \textit{AT2019fdr} or e.g.~\textit{PS1-10adi} is still poorly understood. The TDE interpretation of \textit{AT2019fdr} --- though probably the best one ---  is still only one among others (SLSN Type II or an especially bright `regular' AGN flare). A more extensive study of these objects is needed to provide a definite answer. A potential avenue of research is to obtain the missing host-galaxy spectra for all the accretion flare candidates found in the nuclear sample to \textbf{estimate their corresponding black hole masses}. One could use these to investigate if the host galaxy black hole masses of these events consistently fall below the Hills mass; if they do not, they cannot be TDEs. Also, as already mentioned in Section~\ref{nuc_conclusion}, spectra might help in updating the TDE luminosity function incorporating the new-found transients.

\subsubsection{Real-time application of the Classifier}
Furthermore, a real-time application of a modified version of the classifier would help in \textbf{obtaining early spectroscopy}, maybe also aided by the \textbf{identification of potential dust echoes}. The latter can be a powerful tracer of accretion flares, as was shown in this work. Such a program will also aid in understanding the potential continuum of accretion flares ranging from classical TDEs to the somewhat more ambiguous events like \textit{AT2019fdr}.

So far, we do not know for sure if Tidal Disruption Events or similar violent accretion events do in fact produce high-energy neutrinos. This work, and the associations of a third event in~\cite{Velzen2021} do stipulate that, but they hinge on the correctness of the error regions published by IceCube. Also, since roughly 12 months the follow-up program has not yielded any promising candidates. Ultimately, only time will tell.

\subsubsection{New Neutrino Detectors}
Three important new instruments will allow to shed more light on the origin of high-energy neutrinos. Firstly, \textbf{IceCube-Gen2}~\sidecite{Aartsen2021} is underway, with the goal of being fully operational in the middle of the 2030s. This extended detector will most likely improve upon the localization accuracy of IceCube, rendering associations with sources more secure. Additionally, the eightfold increase in volume when compared to IceCube will yield more events for which an association can be established.

Secondly, the Cubic Kilometre Neutrino Telescope (\textbf{KM3NeT})~\sidecite{AdrianMartinez2016} is currently under construction in the Mediterranean. It is projected to have an angular resolution of \SI{<0.2}{\degree} for neutrino energies above \SI{10}{\tera\eV}~\sidecite{Aiello2019}. This is roughly half an order of magnitude better than the current IceCube accuracy, and will aid in securing associations.

\subsubsection{Rubin Observatory}
Lastly --- and much more imminent --- first light of the \textbf{Rubin Observatory}~\cite{Ivezic2019} is expected for November 2024\sidenote{~\url{https://www.lsst.org/about/project-status}}. With its \SI{8.4}{\m} primary mirror it is by far the largest survey telescope ever built, and will generate an order of magnitude more transient alerts when compared to ZTF. It will have an average sensitivity of 24.5 mag in the \textit{r}-band compared to 20.6 for ZTF, which roughly translates to a 40-fold increase in sensitivity, which would result in a significantly higher number of candidate counterparts.

Therefore, making use of Rubin observatory in IceCube follow up campaigns will \textbf{neccessitate further improvements in photometric typing}. The work done on the nuclear sample might provide a stepping stone for such an undertaking. There are two potential obstacles, though.

Firstly, the lower cadence of Rubin Observatory will lead to a much sparser sampling of light curves when compared to ZTF; this will render photometric typing even harder. In this vein, I have contributed to the HU/DESY participation in the Extended LSST Astronomical Time-series Classification Challenge (ELAsTiCC)~\sidecite{Narayan2023}. This ongoing study for Rubin Observatory consists of photometrically typing simulated Rubin alerts, and the results so far leave me somewhat optimistic.

Secondly, Rubin is located in the southern hemisphere, where IceCube's sensitivity is significantly reduced. It thus seems prudent to additionally keep on relying on smaller sky survey telescopes, even after ZTF ends.

Overcoming these two obstacles is possible, and making use of future instruments will help answering some of the open questions. Whatever those answers will be, one thing is certain: The young field of neutrino astronomy does have an auspicious future.