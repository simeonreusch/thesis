\setchapterimage[7cm]{ztf_telescope.png} % Optionally specify the height
\setchapterpreamble[u]{\margintoc}
\chapter{The Zwicky Transient Facility}
\labch{ZTF}
The second instrument relevant for this thesis is the Zwicky Transient Facility (ZTF). Named after the notorious Swiss-American astronomer Fritz Zwicky, it is a wide-field optical survey telescope located at Mount Palomar in California, United States, at 1700 m above sea level.\marginnote{See \url{https://sites.astro.caltech.edu/palomar/about/telescopes/oschin.html} for a historical overview.} The housing, the 1.22 m (48 inch) Samuel Oschin telescope, is a Schmidt design and was inaugurated in 1948 \cite{Harrington1952}. Originally, the telescope used photographic plates. As these have obvious drawbacks, and because technological progress made it possible, the Near-Earth Asteroid Tracking (NEAT) program \cite{Pravdo1999} replaced the photographic plates with a charge-coupled device (CCD) camera in the early 2000s.

The camera was updated several times over the course of the next years. The immediate predecessor of ZTF, the Palomar Transient Factory (PTF) \cite{Law2009}, began operation in 2009. Equipped with a 96 Megapixel camera, it already had many of the characteristics of ZTF: A fully automated survey, searching for optical transients.

PTF's successor in spirit, ZTF, was the first electronic camera using almost the full field of view (FOV) of the P48. The main design metric for ZTF was \textit{volumentric survey speed} \cite{Bellm2016}. This is the volume within which an object of given absolute magnitude can be detected in one exposure, divided by the total time for the exposure (observation plus overhead). The system saw first light in 2017, and started its scientific use in the year after. As of writing, it is still operational.

\section{Telescope Design}
A Schmidt telescope is 

\section{Camera}

\section{Image Processing}